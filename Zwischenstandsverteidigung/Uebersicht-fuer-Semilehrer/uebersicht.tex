\documentclass[a4paper, 11pt, landscape]{scrartcl}
\usepackage[ngerman]{babel}
\usepackage[utf8]{inputenc}
\usepackage[top=1.5cm, bottom=1cm, left=2.5cm, right=2.5cm, landscape]{geometry}
\usepackage{multicol}
\usepackage{fancyhdr}
\usepackage[hyphens]{url}
\pagenumbering{gobble}

\pagestyle{fancy}
\rhead{\emph{27. Juni 2018}}
\chead{\textsc{Übersicht Zur Zwischenstandsverteidigung}}
\lhead{\emph{Lukas Rost}}

%\usepackage{dblfnote}
%\DFNalwaysdouble

\begin{document}
\centering \textbf{\Large{Entwicklung eines Gamification-basierten Biofeedback-Unterstützungs- und Motivationsgeräts zur Rehabilitation von Schlaganfall-Patienten}}
\\
\begin{multicols*}{2}
\section{Darstellung des Arbeitsstandes}
\begin{itemize}
\item erfolgreiche Fertigstellung der Theoriekapitel und der Kapitel zum Mikrocontroller
\item erfolgreicher Entwurf und Aufbau der Schaltung für den Mikrocontroller
\item erfolgreiche Implementierung eines entsprechend des Konzeptes funktionsfähigen Programms auf dem Mikrocontroller (kann Messdaten vom EMG-Sensor über Bluetooth übertragen)
\item Erstellung eines Feinkonzeptes zu Aufbau und Struktur der Begleit-App
\item Erstellung eines Designprototypen der App (Layout der einzelnen Bildschirmseiten)
\end{itemize}
\section{Kritische Reflexion des Erreichten und der angewandten Arbeitsmethoden}
\begin{itemize}
\item Eventuell wäre eine bessere Ausarbeitung des Konzepts vor der Themenverteidigung notwendig gewesen, da Änderungen nötig waren (u.a. musste das Potentiometer durch einen EMG-Sensor ersetzt werden).
\item Ein schnellerer Arbeitsfortschritt bis zur Zwischenstandsverteidigung wäre wünschenswert gewesen, wurde aber durch langwieriges Literaturstudium und schwierige Fehlersuche im Mikrocontroller-Programm verhindert.
\item Grundsätzlich wurde bisher aber ein guter Arbeitsfortschritt erreicht, der noch innerhalb des Zeitplans liegt. Außerdem wurden alle bisherigen Vorhaben vollständig entsprechend des Konzepts fertiggestellt. Es ist zu erwarten, dass die weiteren Ziele bis zur Abgabe der Seminarfacharbeit erreicht werden können.
\end{itemize}
\section{Darstellung der weiteren Vorhaben}
\begin{itemize}
\item Entwicklung der Android-App bis August 2018
\begin{itemize}
\item[-] Einbindung von Gamification-Elementen wie Erfahrungspunkten und Abzeichen (Badges)
\item[-] dynamische Verteilung sowie Anzeige dieser
\item[-] Anzeige der aktuellen Messwerte mittels tachoähnlicher Anzeige sowie Darstellung in einem Diagramm
\item[-] Erinnerung an durchzuführende Übungen per Benachrichtigungen
\item[-] gegebenenfalls Erweiterungen, z.B. Mehrspielermodus
\end{itemize}
\item Entwicklung des in der App integrierten Minispiels (Zielwerfen mit einem virtuellen Ball) bis Oktober 2018
\item Verfassen der weiteren Kapitel zur Android-App sowie Einleitung und Zusammenfassung bis Dezember 2018
\item Teilnahme am \glqq Jugend forscht\grqq -Wettbewerb 2019
\item gegebenenfalls (sofern sich ein kooperierender Arzt oder Therapeut findet) Praxistest des Geräts an echten Schlaganfallpatienten bis März 2019
\item Vorbereitung und erfolgreiche Durchführung des Kolloquiums bis März 2019
\end{itemize}
\emph{- Erläuterung der Aufgabenverteilung entfällt aus naheliegenden Gründen -}

\end{multicols*}

\end{document}