\subsection{Gamification}
\subsubsection{Grundlegende Mechanismen der Gamification}
Es existieren verschiedene Definitionen für Gamification. Beispielhaft sei hier die Definition nach Breuer zitiert, die besagt, dass es sich bei Gamification um die \glqq Verwendung von spieltypischen Mechaniken außerhalb reiner Spiele, mit dem Ziel, das Verhalten von Menschen zu beeinflussen\grqq \cite{Src:GamifKochOtt} handelt. Entsprechend werden Spielkonzepte verwendet, um die Nutzungsmotivation zu steigern und die Nutzer dazu zu bewegen, mehr oder länger mit einem Produkt zu arbeiten als ohne Gamification. Diese Technik wird innerhalb eines Produktes dazu verwendet, dessen Nutzung zu proklamieren. \cite{Src:GamifKochOtt} \\ \\
Indem spieletypische Merkmale außerhalb spielerischer Zusammenhänge verwendet werden, nutzt man den menschlichen Spieltrieb aus. Es werden positive Anreize gesetzt, um Menschen zu einem bestimmten Verhalten anzuregen, während der Nutzer ebenfalls vorhandene negative Anreize wie eine Bestrafung vermeiden will. \cite{Src:PlanetWissen} Es kommt zu einer \glqq Actio-et-Reactio\grqq -Erfahrung.\cite{Src:GamifKochOtt} Diese Gestaltung und die entsprechende motivationsssteigernde Wirkung lassen sich mithilfe von psychologischen Theorien erklären, auf welche hier jedoch nicht genauer eingegangen werden soll. Gamification an sich ist an sich ein relativ neues Phänomen und wurde auch erst in letzter Zeit umgesetzt. So wurde der Begriff erst 1978 durch Richard Bartle, einen britischen Informatiker und Computerspiel-Pionier, geschaffen. \cite{Src:PlanetWissen} \\ \\
Dieser postulierte auch, dass die Wirkung von Anreizen sich bei verschiedenen Menschen unterscheide. Er entwarf deshalb ein Spielertypen-Koordinatensystem, in dem er Menschen in vier Kategorien verortete, wobei ein Mensch mehreren Kategorien zugleich angehören kann.\cite{Src:Bartle} Im Einzelnen beschrieb er folgende Typen, von denen in einem gelungenen Spiel oder bei einer gelungenen Gamification möglichst viele eine passende Motivation finden sollten:
\begin{itemize}
\item die Erfolgstypen/Achiever, welche im Spiel nach konkreten Maßstäben möglichst viel erreichen wollen.\cite{Src:Bartle} 10 Prozent aller Menschen werden hauptsächlich diesem Typ zugeordnet. \cite{Src:GamifDesign}
\item die Geselligen/Socializer, denen Kontakte und Interaktionen mit Mitspielern, wie sie auch in klassischen Gesellschaftsspielen vorkommen, besonders wichtig sind.\cite{Src:Bartle} 75 Prozent aller Menschen sind vor allem Socializer.\cite{Src:GamifDesign} 
\item die Forscher/Explorer, deren Ziel es ist, möglichst viel zu entdecken und erkunden.\cite{Src:Bartle} Dieser Typ ist bei ca. 10 Prozent aller Menschen am meisten ausgeprägt.\cite{Src:GamifDesign}
\item die Killer, welche Wettbewerb, Wettkampf und Konflikt lieben. Damit sie motiviert sind, müssen andere verlieren und ihnen Respekt erweisen.\cite{Src:Bartle} Dieser Typ macht 5 Prozent aller Menschen aus.\cite{Src:GamifDesign}
\end{itemize}
Um Gamification mit Erfolg anwenden zu können, ist es wichtig, die Kriterien und Mechanismen zu kennen, die ein erfolgreiches Spiel hervorbringen. Wie also kann Gamification angewendet werden, um langweilige, frustrierende, monotone oder unbeliebte Tätigkeiten, wie die dieser Arbeit zugrundeliegende Armlähmungs-Therapie, einfacher und motivierender zu gestalten? Zunächst sollten klare Ziele und Regeln aufgestellt werden, anhand derer der Nutzer weiß, welche Rückmeldung er auf eine bestimmte Aktion bekommt. Diese sogenannte Resultatstransparenz führt zu einer gesteigerten Handlungsmotivation.\cite{Src:GamifKochOtt} Außerdem sollten immer neue Herausforderungen gewährleistet sein.\cite{Src:WwieWissen}\\ \\
Eine motivierende Wirkung erzielt man auch, indem man den Spieler immer oder möglichst leicht gewinnen lässt. Der Spieler sollte beim Spielen in einen \glqq Flow\grqq ~ kommen, der durch eine starke Fokussierung auf das Spiel gekennzeichnet ist.\cite{Src:GamifDesign} Dafür darf ein Spiel weder zu einfach noch zu schwierig sein. Eine Belohnung des Spielers in einem festgelegten Intervall führt zu einer schnellen Abnahme der Motivation. Stattdessen sollte man auf die sogenannte operante Konditionierung setzen, indem man in einem variablen Intervall und in variabler Menge belohnt.\cite{Src:GamifDesign} \\ \\
Ein Spiel kann nach dem MDA-Modell durch drei Bestandteile charaktisiert werden: Mechanics (Spielkomponenten und -funktionen), Dynamics (Interaktion zwischen Spieler und Spiel) und Aesthetics (Emotionen, die beim Spieler erzeugt werden).\cite{Src:GamifDesign} Von diesen kann der Entwickler nur die Mechanics mithilfe spieltypischer Mechanismen beeinflussen. Zu diesen zählen unter anderem:
\begin{itemize}
\item ein Punktesystem gemeinsam mit einer einsehbaren Rangliste (Highscore) bei Mehrspieler-Spielens. Mit diesen können Aktionen des Spielers bewertet werden. Ein Highscore erlaubt einfache Vergleiche durch z.B. eine metrische Punkteskala.\cite{Src:GamifKochOtt} Highscores können auch suggestiv eingesetzt werden, indem sie den Spieler möglichst in der Mitte der Liste zeigen und somit gleichzeitig belohnen und motivieren.\cite{Src:GamifDesign}
\item ein sichtbarer Status durch Titel oder Badges. Diese repräsentieren nach außen, dass der Spieler ein bestimmtes Ziel erreicht hat und bieten somit Vergleichs- und Wettbewerbsmöglichkeiten.\cite{Src:GamifKochOtt} Motivierend sind sie außerdem, da Menschen gerne sammeln. Sie sollten jedoch nicht zu viel eingesetzt werden.\cite{Src:GamifDesign}
\item entdeckbare Aufgaben (Quests), die dem Spiel ein Ziel und Struktur geben und dafür sorgen, dass das  Spiel das Interesse des Spielers behält. Wenn Spieler zur Lösung einer Aufgabe mit Mitspielern zusammenarbeiten müssen, kann dies sozial sehr stark motivieren.\cite{Src:GamifDesign}
\item eine Fortschrittsanzeige, die eine dynamische Visualisierung des bisherigen Erfolgs und noch zu erledigender Aufgaben erlaubt.\cite{Src:GamifKochOtt}
\item gegebenenfalls ein Epic Meaning, also die Arbeit an etwas Erstrebenswerten und an sinnvollen Zielen.\cite{Src:GamifKochOtt}
\end{itemize}
\subsubsection{Beispiele für die motivationssteigernde Wirkung}
Gamification kann erstaunliche Effekte erzeugen. Exemplarisch deutlich wurde das bei einer Reihe von Experimenten der schwedischen Werbeagentur DDB unter dem Namen \glqq The Fun Theory\grqq . Bei einem dieser Experimente wurde eine U-Bahn-Treppe so umgebaut, dass sie wie eine Klaviertastatur bei Bedienung mit den Füßen Töne erzeugte. \footnote[4]{siehe \autoref{fig:pianostairs}} Im Ergebnis wurde diese Treppe sogar öfter als die danebenliegende Rolltreppe benutzt. \cite{Src:PlanetWissen} \\ \\
Auch ein Flaschencontainer, der bunt blinkte, wenn Flaschen in die richtige Öffnung geworfen wurden, und der \glqq tiefste Mülleimer der Welt\grqq , der beim Hineinwerfen von Müll einen Pfiff und einen Aufprall ertönen ließ, stellen erfolgreiche Beispiele dar. Im genannten Mülleimer beispielsweise erhöhte sich die Müllmenge drastisch. \cite{Src:PlanetWissen} Der Spieledesigner Kevin Richardson schlug für dieses Experiment eine Radarfallenlotterie vor, bei der alle, die die Geschwindigkeitsbegrenzung einhielten, an einem Gewinnspiel um die Strafen der Raser teilnahmen. Als dieser Vorschlag umgesetzt wurde, sank die Geschwindigkeit an der kontrollierten Stelle um 20 \% \cite{Src:GamifDesign} \\ \\
Im wissenschaftlichen Umfeld kann Gamification ebenfalls die Motivation steigern und so bessere Ergebnisse erzielen. So gelang es beispielsweise einigen Gamern, mithilfe eines Tetris-Nachbaus innerhalb von zehn Tagen die Proteinstruktur des AIDS-Virus zu entschlüsseln, was die Wissenschaftler hinter dem Projekt 15 Jahre gekostet hätte. \cite{Src:DLFMotiv} Unternehmen versuchen ebenfalls, durch Gamification geeignete Mitarbeiter zu finden. So entwarf der Bayer-Konzern eine Karriere-App im Stil von Quizshows und die Management-Simulation \glqq BIMS Online\grqq , die dort nun zur Rekrutierung geeigneter Fachkräfte dienen. \\ \\
Selbst eine Universität hat Gamification schon ausprobiert: An der Indiana University wurde zeitweise nach einem Experience-Point-System statt nach Noten bewertet. \cite{Src:XPNoten} Ein weiteres Beispiel für gelungene Gamification findet sich bei Frage-Antwort-Websites wie Quora, Stack Exchange oder Stack Overflow, bei welchen andere Nutzer entsprechende Punkte für gute Antworten auf bestimmte Fragen verleihen können. Außerdem kann ein Fragesteller die beste Antwort auf seine Frage markieren. All dies führt dazu, dass die Qualität der Antworten meist konstant hoch ist. \cite{Src:GamifDesign} \\ \\
Die bei solchen Experimenten gewonnenen Beobachtungen legen zumindest empirisch nahe, dass Gamification eine motivationssteigernde Wirkung besitzt. Unter Wissenschaftlern ist dieses Thema jedoch strittig. Einige Studien belegen diese These, andere widersprechen ihr. So wurde in einem Experiment der TU München versucht, das Berufsfeld der Kommissionierung, welches noch relativ wenig automatisiert ist, zu gamifizieren. Im Ergebnis arbeiteten die Kommissionierer schneller und zufriedener als ohne Gamification. \cite{Src:WwieWissen} \\ \\
Bei einer amerikanischen Studie, die sich auf den Erfolg von Fitnessmaßnahmen mit und ohne Wearable fokussierte, kam man jedoch zu gegenteiligen Ergebnissen. Obwohl Wearables ein klassisches Beispiel für Gamification darstellen, war keine signifikant höhere Gewichtsabnahme zu messen. Die mit Wearable ausgestatteten Probanden verloren im Vergleich zur Kontrollgruppe über den Zeitraum von 24 Monaten meist sogar weniger Gewicht. \cite{Src:WearableMotiv}
\subsection{Biofeedback}
In unserem Körper kommt es durch Regulationsvorgänge ständig zu Veränderungen von messbaren Zustandsgrößen bei biologischen Vorgängen und Körperfunktionen. Diese sind, der unmittelbaren Sinneswahrnehmung, also dem Bewusstsein, nicht zugänglich. Sie können jedoch mit technischen und elektronischen Hilfsmitteln beobachtbar gemacht werden, was fachsprachlich unter dem Begriff Biofeedback bekannt ist. \cite{Src:BiofeedWiki} Oft wird dieses Verfahren zur Rehabilitation von erlahmten Muskeln, wie z.B. bei einer Armlähmung, eingesetzt. \\ \\
Da solche Körperfunktionen dem Bewusstsein normalerweise nicht zugänglich sind, können sie auch nicht beeinflusst werden. Durch Biofeedback wird genau das jedoch möglich. Durch die Visualisierung der Messwerte für den Patienten kommt es zu einer Rückkoppelung und dieser kann Kontrolle über die Körperfunktion ausüben. Weiterhin können über die operante Konditionierung ganze Reiz-Reaktions-Muster erlernt werden. Insgesamt wird eine Bewusstseinsschärfung für die eigenen inneren Zustände erreicht und die Einflussnahme auf das Nervensystem somit vereinfacht. \cite{Src:BiofeedWiki} \\ \\
Die Messwerte können nicht nur visualisiert werden, sondern werden bei manchen Anwendungen auch als Töne dargestellt. Die oft kleinen und tragbaren Messgeräte, die beim Biofeedback benutzt werden, messen meist nichtinvasiv. \cite{Src:BiofeedWiki} Ihre Messergebnisse werden dann in einem Analog-Digital-Wandler konvertiert, gemittelt und verstärkt. Anschließend werden sie per kabelloser Bluetooth-Übertragung auf ein zur Darstellung geeignetes Gerät übertragen. \\ \\
Grundsätzlich können per Biofeedback viele Größen gemessen werden, wie beispielsweise Atem, Blutdruck, Blutwerte, der Hautwiderstand oder Gehirnströme. Im hier beschriebenen Anwendungsfall entschied ich mich jedoch für die Messung von Muskelpotentialen per Elektromyografie, da mit dieser die Beweglichkeit des Arms am besten erfasst werden kann. In vielen Fällen erzielt Biofeedback positive Ergebnisse, sodass es in manchen Fällen sogar eine Alternative zu Medikamenten sein kann. \cite{Src:BiofeedWiki}