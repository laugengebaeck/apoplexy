Das im Rahmen dieser Arbeit entstandene Gerät mitsamt der App soll Patienten bei der Therapie von Armlähmungen unterstützen und motivieren. Deshalb waren die Möglichkeit zur Durchführung von Bewegungsübungen und ein durch solche steuerbares Minispiel Bestandteile der Zielstellung. \\ \\
Um diese in eine App zu integrieren und diese möglichst motivierend zu gestalten, kam das Prinzip der Gamification zum Einsatz. Dabei werden spieltypische Elemente wie Ranglisten, Erfahrungspunkte und Aufgaben außerhalb spielerischer Zusammenhänge angewendet. Mithilfe von Gamification lässt sich das Verhalten von Menschen beeinflussen, in diesem Fall etwa, um die Motivation der Patienten zu erhöhen. Damit Gamification erfolgreich ist, sollte man verschiedene Kriterien beachten, die im Idealfall ein erfolgreiches Spiel hervorbringen. Dazu zählt beispielsweise das Spielertypen-Koordinatensystem nach Bartle. \\ \\
Gamification kann, wie das Experiment \glqq The Fun Theory\grqq ~zeigte, motivationssteigernde Wirkung haben, auch wenn dies wissenschaftlich umstritten ist. Weiterhin wird in der App das Konzept des Biofeedbacks umgesetzt, das Körperfunktionen mithilfe von technischen Mitteln für das Bewusstsein beobachtbar macht. \\ \\
Zur Therapie von Schlaganfällen existieren verschiedene Methoden. Nach ersten Maßnahmen wie der Stabilisierung der Vitalfunktionen und einer Thrombolyse-Therapie müssen die aufgetretenen Folgen behandelt werden. Dazu existieren verschiedene Therapiekonzepte wie das Bobath-Konzept, die versuchen, verlorene Fähigkeiten zu kompensieren. Dabei werden Methoden wie das Arm-Basis- und das Arm-Fähigkeits-Training angewendet, die als Grundlage für das in dieser Arbeit entwickelte Gerät dienen. \\ \\
Das Gerät selbst besteht aus einer Schaltung. Diese enthält einen Mikrocontroller, einen Muskelsensor (EMG) und einen Bluetooth-Chip. Dabei nimmt der Mikrocontroller die durch den Sensor gelieferten Messdaten auf und sendet sie an ein in der Nähe befindliches Smartphone. Dieser Ablauf wurde mithilfe eines Programms in C realisiert.\\ \\
Auf diesem Smartphone befindet sich eine App für das Betriebssystem Android, die mit dem Gerät über Bluetooth kommuniziert und die Messwerte auswertet. Dabei erlaubt sie es, Bewegungsübungen durchzuführen, während diese grafisch dargestellt werden können. Ein Minispiel, welches als Android-\textit{View} umgesetzt wurde und ähnlich wie das bekannte Spiel \textit{Flappy Bird} funktioniert, ist ebenfalls vorhanden. Beide Bereiche werden über ein Gamification-System verbunden, das aus je nach Leistung verteilten Erfahrungspunkte und Abzeichen bzw. Aufgaben besteht. Diese Aufgaben werden in einer SQL-Tabelle gespeichert. Weiterhin kann die App mithilfe von Benachrichtigungen an Übungen erinnern. \\ \\
Es bestehen einige Weiterführungsmöglichkeiten. So wäre es zum Beispiel möglich, Mehrspielerfunktionen hinzuzufügen, die dem Spiel eine soziale Komponente geben würden. Sollte das Gerät tatsächlich eingesetzt werden, so wäre es sinnvoll, zusammen mit Ärzten ein tragfähiges Therapiekonzept zu entwickeln. Auch könnte man mithilfe professioneller Spieledesigner die Gamification ausbauen. Ein genauerer Sensor sowie eine Übersicht über den Therapieerfolg für den Arzt wären in dieser Hinsicht ebenfalls mögliche Erweiterungen. \\ \\
Abschließend lässt sich sagen, dass das hier vorliegende Gerät samt der App bei entsprechender Entwicklung einiges Potential birgt. So meinte sogar der Robotiker Sami Haddadin in einem Interview zu Robotern: \glqq Die Rehabilitation [ist] ein großes Thema, also das Bewegungstraining bei Schlaganfallpatienten [...]\grqq\cite{Src:CTHaddadin} Ein solches Gerät könnte dabei vielleicht schon in naher Zukunft eine wichtige Rolle spielen.