\renewcommand{\refname}{Literatur- und Quellenverzeichnis}

\begin{thebibliography}{xxxxxx}

\subsection*{Literaturquellen}

\bibitem[1] {Src:GamifDesign} Cunningham, Christopher; Zichermann, Gabe: \textit{Gamification by Design} - Implementing Game Mechanics in Web and Mobile Apps, 1. Auflage, Sebastopol, O'Reilly Verlag, 2011, \url{https://doc.lagout.org/programmation/Game Design/Gamification by Design - Zichermann, Cunningham - O'Reilly (2011)/Gamification by Design - Zichermann, Cunningham - O'Reilly (2011).pdf}

[Zugriff am 30.1.2018, 18:20 Uhr]

\bibitem[2] {Src:AndroidGargenta} Gargenta, Marko: \textit{Einführung in die Android-Entwicklung}, 1. Auflage, Köln, O'Reilly Verlag, 2011

\bibitem[3] {Src:CTHaddadin} Grävemeyer, Arne: \textit{Ära starker Bots mit zarten Fingern}, Roboterforscher Haddadin erwartet Wandel in Industrie und Haushalt, in: c't 11/2018, S. 68 - 69

\bibitem[4] {Src:Kotlin} Isakova, Svetlana; Jemerov, Dmitry: \textit{Kotlin in Action}, 1. Auflage, Shelter Island, Manning Publications, 2017

\bibitem[5] {Src:WearableMotiv} Jakicic, John M. et al.: \textit{Effect of Wearable Technology Combined With a Lifestyle Intervention on Long-term Weight Loss}, The IDEA Randomized Clinical Trial, In: Journal of the American Medical Association, 316(11)/2016, S. 1161 - 1171, \url{https://jamanetwork.com/journals/jama/articlepdf/2553448/joi160104.pdf}

[Zugriff am 27.12.2017, 15:37 Uhr]

\bibitem[6] {Src:AndroidKuenneth} Künneth, Thomas: \textit{Android 8 - Das Praxisbuch für Java-Entwickler}, 5. aktualisierte Auflage, Bonn, Rheinwerk Verlag, 2018

\bibitem[7] {Src:DGNeurorehab} Platz, Thomas; Roschka, Sybille: \textit{Rehabilitative Therapie bei Armlähmungen nach einem Schlaganfall}, Patientenversion der Leitlinie der Deutschen Gesellschaft für Neurorehabilitation, Bad Honnef, Hippocampus Verlag, 2011, 
\url{http://www.kompetenznetz-schlaganfall.de/fileadmin/download/Arm-Reha/Leitlinie_Therapie_Armlaehmung_220911-verlinkt.pdf} 

[Zugriff am 27.12.2017, 16:04 Uhr]

\bibitem[8] {Src:SchmittAVR} Schmitt, Günter: \textit{Mikrocomputertechnik mit Controllern der Atmel AVR-RISC-Familie}, Programmierung in Assembler und C - Schaltungen und Anwendungen, 4. Auflage, München, Oldenbourg Verlag, 2008

\subsection*{Internetquellen}

\bibitem[9] {Src:ApoFlex} Antwerpes, Frank et al.: \textit{Schlaganfall}, \url{http://flexikon.doccheck.com/de/Schlaganfall}

[Zugriff am 31.12.2017, 12:35 Uhr]

\bibitem[10] {Src:AtmelDBeins} Atmel Corporation: \textit{ATmega48A/PA/88A/PA/168A/PA/328/P}, Atmel 8-Bit Microcontroller with 4/8/16/32 KBytes In-System Programmable Flash, Datasheet, \url{http://www.atmel.com/images/Atmel-8271-8-bit-AVR-Microcontroller-ATmega48A-48PA-88A-88PA-168A-168PA-328-328P_datasheet_Complete.pdf} 

[Zugriff am 27.12.2017, 11:46 Uhr]

\bibitem[11] {Src:AtmelDBzwei} Atmel Corporation: \textit{ATmega48PA/88PA/168PA}, 8-bit AVR Microcontrollers, Datasheet Complete, \url{http://ww1.microchip.com/downloads/en/DeviceDoc/Atmel-42734-8-bit-AVR-Microcontroller-ATmega48PA-88PA-168PA_Datasheet.pdf} 

[Zugriff am 27.12.2017, 11:48 Uhr]

\bibitem[12] {Src:AVRTutor} Autorengemeinschaft: \textit{AVR-GCC-Tutorial}, \url{https://www.mikrocontroller.net/articles/AVR-GCC-Tutorial}

[Zugriff am 27.12.2017, 12:10 Uhr]

\bibitem[13] {Src:Bartle} Autorengemeinschaft: \textit{Bartle-Test}, \url{https://de.wikipedia.org/wiki/Bartle-Test}

[Zugriff am 4.4.2018, 14:11 Uhr]

\bibitem[14] {Src:BiofeedWiki} Autorengemeinschaft: \textit{Biofeedback}, \url{https://de.wikipedia.org/wiki/Biofeedback}

[Zugriff am 5.4.2018, 16:25 Uhr]

\bibitem[15] {Src:ApoWiki} Autorengemeinschaft: \textit{Schlaganfall}, \url{https://de.wikipedia.org/wiki/Schlaganfall}

[Zugriff am 31.12.2017, 13:01 Uhr]

\bibitem[16] {Src:PlanetWissen} Drescher, Frank: \textit{Spiele und Spielzeug - Gamification}, \url{https://www.planet-wissen.de/gesellschaft/spiele_und_spielzeug/gamification/index.html}

[Zugriff am 30.3.2018, 10:12 Uhr]

\bibitem[17] {Src:ApoNetdokt} Feichter, Martina: \textit{Schlaganfall}, \url{https://www.netdoktor.de/krankheiten/schlaganfall/}

[Zugriff am 31.12.2017, 13:25 Uhr]

\bibitem[18] {Src:BobathFlex} Freyer, Timo; Mörkl, Sabrina; Ostendorf, Norbert: \textit{Bobath-Konzept}, \url{http://flexikon.doccheck.com/de/Bobath-Konzept}

[Zugriff am 3.1.2018, 17:42 Uhr]

\bibitem[19] {Src:BluetoothHC} ITead Studio: \textit{HC-05}, Bluetooth to Serial Port Module, \url{http://www.electronicaestudio.com/docs/istd016A.pdf}

[Zugriff am 27.12.2017, 11:13 Uhr]

\bibitem[20] {Src:GamifKochOtt} Koch, Michael; Ott, Florian: \textit{Gamification} – Steigerung der Nutzungsmotivation durch Spielkonzepte, Projekt mit der Forschungsgruppe Kooperationssysteme an der Universität der Bundeswehr München, \url{http://www.soziotech.org/gamification-steigerung-der-nutzungsmotivation-durch-spielkonzepte/}

[Zugriff am 30.1.2018, 18:03 Uhr]

\bibitem[21] {Src:XPNoten} Parrish, Kevin: Professor Uses RPG-like Exp Rather Than Grades, \url{https://www.tomsguide.com/us/Experience-Points-XP-Indiana-University,news-6183.html}

[Zugriff am 5.4.2018, 15:32 Uhr]

\bibitem[22] {Src:RehabNelles} Nelles, Gereon et al.: \textit{Motorische Rehabilitation nach Schlaganfall}, \url{http://www.friedehorst.de/nrz/rehabilitation.pdf?m=1140520827}

[Zugriff am 27.12.2017, 16:16 Uhr]

\bibitem[23] {Src:DLFMotiv} Reinhardt, Anja: \textit{Motivation und Manipulation im Alltag}, Spieltheorie \glqq Gamification\grqq , \url{http://www.deutschlandfunk.de/spieltheorie-gamification-motivation-und-manipulation-im.724.de.html}

[Zugriff am 5.4.2018, 15:59]

\bibitem[24] {Src:EMGdetect} Seeed Technology Co. Ltd.: \textit{Grove - EMG Detector}, \url{http://wiki.seeed.cc/Grove-EMG_Detector/}

[Zugriff am 30.12.2017, 14:03 Uhr]

\bibitem[25] {Src:Destatis} Statistisches Bundesamt: \textit{Ergebnisse der Todesursachenstatistik für Deutschland 2015}, ausführliche 4-stellige ICD10-Klassifikation, \\ \url{https://www.destatis.de/DE/Publikationen/Thematisch/Gesundheit/Todesursachen/Todesursachenstatistik5232101157015.xlsx?__blob=publicationFile}

[Zugriff am 2.1.2018, 11:16 Uhr]

\bibitem[26] {Src:WwieWissen} W wie Wissen: \textit{Gamification - Wie Spielen den Alltag interessanter macht}, Ausschnitt aus der gleichnamigen Sendung in Das Erste am 19.12.2015 (16.00 Uhr), \url{http://www.ardmediathek.de/tv/W-wie-Wissen/Gamification-Wie-Spielen-den-Alltag-in/Das-Erste/Video?bcastId=427262&documentId=32368232}

[Zugriff am 4.4.2018, 14:32 Uhr]

\subsection*{Bildquellen}

\bibitem[\autoref{fig:fasttest}]{Abb1} \url{http://www.volkskrankheiten.at/images/1237/widgets/Schlaganfall-(2).svg}

[Zugriff am 31.12.2017, 13:37 Uhr]

\bibitem[\autoref{fig:pianostairs}]{Abb2} \url{https://scontent-frx5-1.xx.fbcdn.net/v/t1.0-9/225807_10150179024982659_311531_n.jpg?_nc_cat=0&oh=986b13c0619d22e40dc9de883a6ed930&oe=5B5CEE4C}

[Zugriff am 17.5.2018, 15:56 Uhr]
\end{thebibliography}
