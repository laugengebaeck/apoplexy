\glqq [Der Schlaganfall ist] nach Herzerkrankungen und Krebsleiden [...] die dritthäufigste Todesursache in Deutschland.\grqq\cite{Src:ApoFlex} So gab es im Jahr 2015 rund 40000 Todesfälle durch einen Schlaganfall oder seine Folgen.\cite{Src:Destatis} Doch viele der Betroffenen überleben glücklicherweise, tragen dabei jedoch Langzeitbehinderungen davon. So ist er sogar die häufigste Ursache für diese. \\ \\
Für die davon betroffenen Patienten bedeutet das langwierige Bewegungsübungen, bei denen der Erfolg noch nicht einmal sicher ist. In Studien wurde beispielsweise festgestellt, dass nur 5 Prozent der Patienten nach einer solchen Therapie wieder in der Lage waren, ihre Arme und Hände uneingeschränkt einzusetzen, während bei 20 Prozent der Patienten keine Hand- und Armfunktion zurückkehrte.\cite{Src:RehabNelles} \\ \\
Um eine solche Rehabilitation so erfolgreich wie möglich zu gestalten, existieren verschiedene Methoden, die in dieser Arbeit auch kurz vorgestellt werden sollen. Ihnen ist jedoch eines gemeinsam: Sie legen wenig Wert darauf, den Patienten zu unterstützen, zu motivieren und möglicherweise den Heilungsprozess durch solche positiven Wirkungen zu vereinfachen und zu beschleunigen. Dabei gibt es Methoden, die in dieser Hinsicht vielversprechend klingen, zum Beispiel die Gamification, auch Spielifizierung genannt. Bei dieser werden spieltypische Elemente in fremde Kontexte wie hier eine Therapie integriert, wodurch motivierende Wirkungen erzielt werden können. \\ \\
Ausgehend von dieser Überlegung soll in dieser Arbeit ein Gerät entwickelt werden, das bei der Therapie einer Armlähmung unterstützend wirken kann. Es soll Patienten motivieren, unterstützen und gegebenenfalls Hinweise bezüglich des Behandlungsfortschritts geben. Das Gerät benutzt dazu einen Elektromyografie-Sensor, der mittels Oberflächenelektroden die Kontraktion der Armmuskeln messen kann. Über Bluetooth-Funk werden diese Sensordaten an ein Smartphone übertragen, das der Patient mit einer zugehörigen Begleitapp benutzen kann. \\ \\
Diese App nutzt Methoden der Gamification wie Erfahrungspunkte oder Aufgaben (\glqq Quests\grqq), um die Motivation des Patienten bei solch monotonen Übungen zu steigern. Bezogen auf den Aufbau soll sie ähnliche wie eine konventionelle Fitness-App die Möglichkeit bieten, Übungen durchzuführen und auszuwerten. Nicht zuletzt soll sie es auch erlauben, ein Minispiel zu spielen, welches durch Armbewegungen gesteuert werden kann. \\ \\
Um die Motivation des Patienten über längere Zeit zu erhalten, sollen Benachrichtigungen den Nutzer, falls gewünscht, täglich an seine Übungen erinnern. Doch da die beste technische Lösung ohne die Einbindung in eine medizinisch anerkannte Therapie sinnlos ist, soll diese Arbeit auch auf Möglichkeiten dazu eingehen. \\ \\
Besonders danken möchte ich meinem Fachbetreuer Herrn Johannes Süpke, der mich im Prozess der Erstellung der Arbeit sowohl bei inhaltlichen und fachlichen Fragen als auch bei der Bereitstellung der nötigen Hardware unterstützte. Weiterhin gilt mein Dank meiner Seminarfachbetreuerin Frau Dr. Marion Moor für die Unterstützung in formalen, rhetorischen und sprachlichen Fragen. \\ \\
Außerdem möchte ich meinem Außenbetreuer Herrn Hannes Weichel danken, der mich mit wichtigen Vorschlägen zu Thema und Konzept der Arbeit sowie mit der Bereitstellung eines Mikrocontrollers und weiterer Materialien unterstützte. Nicht unerwähnt bleiben sollen jedoch auch Herr Frank Paulig und das SFZ Erfurt sowie Herr Udo Weitz und der Sponsorpool Thüringen des Wettbewerbs \glqq Jugend forscht\grqq , die die nötigen finanziellen Mittel für die Beschaffung noch nicht vorhandener Hardware bereitstellten.