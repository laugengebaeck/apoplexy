\subsection{Schlaganfall als Krankheitsbild}
Beim Schlaganfall, auch \textit{Apoplexia cerebri} genannt, handelt es sich allgemein um eine \glqq plötzliche Durchblutungsstörung im Gehirn.\grqq ~\cite{Src:ApoNetdokt} Durch diese kommt es zu einem regionalen Mangel an Sauerstoff und Nährstoffen, welcher zu einem Absterben von Gehirngewebe führt.
\\ \\
Es existieren zwei mögliche Ursachen: Der ischämische oder Hirninfarkt tritt bei 80 bis 85 \% der Fälle auf. In diesem Fall ergibt sich eine mangelnde Durchblutung aufgrund von Gefäßverschlüssen, auch unter dem Begriff Arteriosklerose oder Thrombose zusammengefasst. Folgend kann es dabei zusätzlich auch zu einem Schlaganfall der zweiten Art kommen, dem sogenannten hämorrhagischen Infarkt bzw. der Hirnblutung, die 10 bis 15 \% der Fälle zugrundeliegt. Dieser wird durch geplatzte und eingerissene Gefäße verursacht, aus denen Blut ins Hirngewebe austritt. Dieses schädigt durch eine verminderte Sauerstoffversorgung, seinen Druck sowie seine neurotoxische Wirkung das Gehirn. Ein solcher Infarkt kann seinerseits wiederum eine Ischämie verursachen.\cite{Src:ApoFlex}
\\ \\
Im Vorfeld eines Schlaganfalls treten oft transitorisch-ischämische Attacken, also vorübergehende neurologische Ausfälle, auf. Symptome eines Schlaganfalls reichen von halbseitigen Körperlähmungen über Sprachstörungen und eingeschränktes Sprachverständnis (motorische und sensorische Aphasie) bis zu Sehstörungen und Gleichgewichtsproblemen. Auch Verwirrtheit, Übelkeit und Kopfschmerzen können auftreten.\cite{Src:ApoFlex}
\\ \\
Zur Erkennung von Schlaganfällen wird meist der FAST-Test benutzt, der fachsprachlich unter dem Begriff \glqq Cincinnati Prehospital Stroke Scale\grqq ~bekannt ist.\cite{Src:ApoWiki} Dieser besteht aus folgenden Punkten:
\begin{enumerate}
\item \textbf{F}ace/Gesicht: Person kann nur mit einer Gesichtshälfte lächeln
\item \textbf{A}rms/Arme: Unfähigkeit, beide Arme mit nach oben geöffneten Handflächen nach vorne zu strecken
\item \textbf{S}peech/Sprache: undeutliche Aussprache, kann nicht sprechen, versteht nichts mehr
\end{enumerate}
Sollten diese Punkte zutreffen, zählt der vierte Punkt: \textbf{T}ime (Zeit). Der Rettungsdienst sollte umgehend verständigt werden, um den Schaden zu begrenzen, denn \glqq Zeit ist Hirn \grqq.\cite{Src:ApoFlex}\footnote[2]{siehe \autoref{fig:fasttest}}
Die Diagnostik erfolgt in einer geeigneten Klinik mit \glqq Stroke Unit\grqq .~Dabei werden meist Verfahren wie Computertomografie und Magnetresonanztomografie genutzt. Die Schwere des Schlaganfalls wird mit Scoresystemen wie der \glqq National Institutes of Health Stroke Scale\grqq ~beurteilt.
\\ \\
Risikofaktoren umfassen unter anderem Bluthochdruck, Rauchen, Diabetes sowie Übergewicht und Bewegungsmangel. Dementsprechend wirkt eine gesunde Lebensweise präventiv. Dazu gehören gesunde Ernährung und regelmäßige Bewegung sowie ausreichende Flüssigkeitsaufnahme, aber auch Stressvermeidung.\cite{Src:ApoFlex} Es existieren jedoch auch andere, nicht beeinflussbare Risikofaktoren wie Alter, Blutgruppe und genetische Veranlagung.
\\ \\
Zu den Basismaßnahmen bei der Therapie gehört als erstes die Stabilisierung der Vitalfunktionen wie Blutdruck, Puls und Körpertemperatur. Auch sollte der Patient mit erhöhtem Oberkörper gelagert werden. diese Maßnahmen stabilisieren den Patienten und dienen der Verhinderung eines weiteren Schlaganfalls. Weiterhin kann bei einem ischämischen Infarkt bis zu drei Stunden nach Auftreten des Schlaganfalls abhängig von der Größe des Infarkts eine intravenöse Thrombolyse-Therapie durchgeführt werden, um eventuell verschlossene Blutgefäße wieder zu öffnen. Bei Hirnblutungen dagegen sind operative Behandlungen sinnvoll, beispielsweise zur Hirndruck-Entlastung.\cite{Src:ApoFlex}
\subsection{Rehabilitation und Effektivität von Bewegungsübungen}
Armlähmungen zählen zu den häufigsten Folgen einer Hirnschädigung, wie sie durch einen Schlaganfall hervorgerufen wird.\cite{Src:DGNeurorehab} Aber auch Lähmungen anderer Körperteile oder Aphasie können auftreten.\cite{Src:ApoWiki} Meist betrifft eine solche Hirnschädigung nur eine Gehirnhälfte, sodass es nur auf einer Körperseite zu Lähmungen kommt. \\ \\
Die meisten Rehabilitationsmaßnahmen dienen dazu, die Körperwahrnehmung des Patienten zu fördern und verlorene Fähigkeiten zu kompensieren. Dabei sind Ansätze wie die \glqq Constraint-Induced Movement Therapy\grqq ~vielversprechend. Bei dieser auch als \glqq Taubsche Bewegungsinduktion\grqq ~bekannten Methode wird der gesunde Arm täglich über längere Zeit immobilisiert und der Betroffene somit gezwungen, die erkrankte Hand zu benutzen.\cite{Src:ApoWiki} So kann ein \glqq erlernter Nichtgebrauch\grqq ~verhindert werden. \cite{Src:RehabNelles}\\ \\
Ein weiteres bekanntes Rehabilitationskonzept ist das Bobath-Konzept, welches annimmt, dass gesunde Hirnregionen die Aufgaben geschädigter Hirnregionen übernehmen können. Durch das Konzept soll eine entsprechende Vernetzung innerhalb des Gehirns gefördert und die vom Schlaganfall betroffene Körperseite wieder in Bewegungen einbezogen werden.\cite{Src:BobathFlex} Solche Ansätze erfordern interdisziplinäre Zusammenarbeit. So können beispielsweise mit Physiotherapeuten Gangmuster eingeübt werden, während Ergotherapeuten an der Wiederherstellung der sensomotorischen Fähigkeiten arbeiten und Logopäden mit Sprachtherapie die Aphasie behandeln.\cite{Src:ApoWiki} \\ \\
Armlähmungen als solches zeigen sich unter anderem in einer stark beeinträchtigten willentli-
chen Bewegungsfähigkeit, aber auch durch \glqq  erhöhte Muskelanspannung (,Spastik') mit einer Fehlstellung des Armes in Ruhe\grqq ~sowie der Schwierigkeit, den Arm passiv zu bewegen. \cite{Src:DGNeurorehab} Zunächst sollten Armlähmungen medizinisch beurteilt werden, wozu spezielle Verfahren wie der Fugl-Meyer-Test existieren. Auf diese soll hier jedoch nicht genauer eingegangen werden. \\ \\
Es existieren verschiedene Therapiemethoden mit und ohne Technikeinsatz.
Das bilaterale Training beispielsweise besteht darin, dass mit beiden Armen gleichzeitig symmetrische Bewegungen ausgeführt werden, sodass in beiden Armen eine gleichmäßige Bewegungsfähigkeit hergestellt wird. Es wird in verschiedenen Studien neutral bis positiv beurteilt.\cite{Src:DGNeurorehab} \\ \\
Das schädigungsorientierte Training zielt darauf ab, spezifische Behinderungen bei alltäglichen Tätigkeiten zu beheben. Es existieren zwei Formen. Das Arm-Basis-Training beübt alle Bewegungsmöglichkeiten des Arms, also Schulter, Ellenbogen, Handgelenk und Finger. Es ist dabei auf Patienten mit schweren Lähmungen ausgelegt.\cite{Src:DGNeurorehab} Das Arm-Fähigkeits-Training dagegen schult verschiedene Formen von Geschicklichkeit und wird bei leichten Lähmungen angewendet.\cite{Src:DGNeurorehab}Beide Formen dieses Trainings zeigen einen positiven Effekt.\cite{Src:RehabNelles} Das aufgabenorientierte Training stellt eine weitere Trainingsform dar, bei der über Bewegungsaufgaben aus dem Alltag die funktionellen Fähigkeiten des Arms wiederhergestellt werden sollen. Es wird neutral beurteilt.\cite{Src:DGNeurorehab}\\ \\
Doch auch ein Technikeinsatz kann bei einer solchen Therapie erfolgen. So existiert beispielsweise die Armrobot-Methode, bei der ein Roboter nicht selbständig ausführbare Bewegungen mechanisch unterstützt. In Bezug auf die Effektivität wird diese positiv beurteilt.\cite{Src:DGNeurorehab} Auch die neuromuskuläre Elektrostimulation ist eine mögliche Therapie, bei der ein Gerät per Elektromyographie Bewegungsversuche des Muskels erkennt und diesen daraufhin elektrisch stimuliert, was zu einer großen Bewegung führt.\cite{Src:DGNeurorehab} Die Therapie wurde neutral beurteilt.\cite{Src:RehabNelles} \\ \\
Ausgehend von diesen Informationen entschied ich mich dazu, das Gerät als Unterstützung für Arm-Basis- und Arm-Fähigkeits-Training zu konzipieren. Im Zuge einer Erweiterung könnten auch Ansätze wie Armrobot und Elektrostimulation einbezogen werden.